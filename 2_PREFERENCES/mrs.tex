\section{Marginal Rate of Substitution}


\subsection{Marginal Rate of Substitution}

\slide{Marginal Rate of Substitution (MRS)}{
    \dfn{The slope of an indifference curve is known as the marginal rate of substitution
    (MRS).}

    This is a measure of the rate at which the consumer is just willing to substitute one
    good for another.

    We take a little of good 1, \(\Delta x_1\), away from the consumer. Then we give him
    \(\Delta x_2\), an amount that is just sufficient to put him back on his indifference curve.

    As \(\Delta x_1\) gets smaller, so does \(\Delta x_2\). When these become infinitesmial,
    the ratio \(\frac{\Delta x_2}{\Delta x_1}\) approaches the slope of the indifference curve.

    Monotonic preferences imply that the indifference curves have a negative slope. Therefore,
    the MRS is typically a negative number.
}


\slide{MRS Graphically}{
    \begin{center}
        \includegraphics[scale=0.4]{MRS1}
    \end{center}
}


\subsection{Diminishing Marginal Rate of Substitution}

\slide{Diminishing MRS}{
    From the previous figure (in slide 42), observe that the MRS varies along the indifference curve.

    Think about MRS as the amount of good \(x_2\) a consumer is willing to give up for one unit extra
    unit of good \(x_1\).

    We can think about this intuitively:
    \begin{itemize}
        \item If I have a low amount of good \(x_1\) and high amount of good \(x_2\), I would ideally
        want more of \(x_1\) (because of convex preferences). Therefore, I will give up a lot of \(x_2\)
        for some \(x_1\).
        \item If I have a lot of \(x_1\) and less of \(x_2\), then I am willing to give up a lower
        amount of \(x_2\).
    \end{itemize}
}


\section{Summary}

\slide{Summary}{
    \begin{itemize}
        \item A consumer's preferences can be represented by indifference curves.
        \item An indifference curve is the set of all bundles that the consumer is indifferent between.
        \item We assume well-behaved preferences which include monotonicity and convexity.
        \item The slope of an indifference curve is called the marginal rate of substitution (MRS).
    \end{itemize}
}


\section{References}

\slide{References}{
    Varian, H. R. (2014). Intermediate Microeconomics: A Modern Approach (9th ed.). W. W. Norton \& Company.
}