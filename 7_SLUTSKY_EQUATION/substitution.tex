\section{Introduction}

\slide{What is the Slutsky Equation About?}{
  We want to consider how a consumer's choice of a good responds to changes in its price.

  It is natural to think that when the price of a good rises the demand for it will fall. However,
  Giffen goods are an exception to this rule.
  Giffen goods are pretty peculiar and are primarily a theoretical curiosity, but they do exist in the real world.
  
  For example, in the 19th century, potatoes were considered a Giffen good in Ireland during the Great Famine.
  As the price of potatoes rose, people could not afford to buy more expensive foods and ended up consuming more
  potatoes, which were a staple food.

  What is going on here? How is it that changes in price can have these ambiguous effects on demand?
  Here, we'll try to sort out these effects.
}


\slide{What is the Slutsky Equation About?}{
  The Slutsky equation is a mathematical decomposition of the total effect on demand of a price change into two components:
  \begin{enumerate}
    \item Substitution Effect
    \item Income Effect
  \end{enumerate}
}


\section{Substitution Effect}


\subsection{Introduction}

\slide{Substitution Effect}{
  When the price of a good changes, there are two sorts of effects: the rate
  at which you can exchange one good for another changes, and the total
  purchasing power of your income is altered.

  The first part --- the change in demand due to the change in the rate
  of exchange between the two goods --- is called the substitution effect.

  In order to give a more precise definition we have to consider
  the effect in greater detail.
}


\subsection{Pivot and Shift}

\slide{Pivot and Shift}{
  Consider the following diagram:
  \begin{center}
    \includegraphics[scale=0.4]{SE1.png} \\
    \small{Figure 1: Pivot and Shift}
  \end{center}
}


\slide{Pivot and Shift}{
  Here we have a situation where the price of good 1 has declined. This means that the budget
  line rotates outwards about the vertical intercept \(\frac{m}{p_2}\) and becomes flatter.

  We can break this movement of the budget line up into two steps: first \emph{pivot}
  the budget line around the \emph{original demanded bundle} and then \emph{shift} the
  pivoted line out to the \emph{new demanded bundle}.

  This “pivot-shift” operation gives us a convenient way to decompose
  the change in demand into two pieces. 
  
  The first step --- the pivot --- is a movement where the slope of the budget line changes while its purchasing
  power stays constant, while the second step --- the shift --- is a movement where the slope
  stays constant and the purchasing power changes.
}


\slide{Pivot and Shift}{
  The purchasing power of the consumer has remained constant in the sense that the original bundle of goods is just affordable at
  the new pivoted line.

  This neccessitates the calculation of how much we have to change income by in order to keep the old bundle affordable at the new price.

  Let \(m'\) be the new income associated with the pivoted budget line. Since \((x_1,x_2)\) is affordable at both \((p_1,p_2,m)\)
  and \((p_1',p_2,m')\), we have:
  \[m' = p_1' x_1 + p_2 x_2\]
  \[m = p_1 x_1 + p_2 x_2\]
  
  Then, \(m' - m\) will be the change in income necessary to make the old bundle affordable at the new prices.
}


\slide{Pivot and Shift}{
  \[m' - m = x_1(p_1' - p_1)\]

  Letting \(\Delta p_1 = p_1' - p_1\) represent the change in price of good 1, and \(\Delta m = m' - m\) represent the change in money income, we have:
  \begin{equation}
    \Delta m = x_1 \Delta p_1
  \end{equation}

  Note that the change in income and the change in price will always move
  in the same direction: if the price goes up, then we have to raise income to
  keep the same bundle affordable. This needs more elaboration.
}


\slide{Pivot and Shift}{
  If the price of good 1 decreases, as shown in Figure 1, then the consumer can afford to buy more of good 1 with the same amount of money.

  However, this new budget line will not pass through the original bundle.
  
  To make the new budget line pass through the original bundle,
  we need to reduce the consumer's income by an amount equal to \(x_1 \Delta p_1\).

  This is what is we mean when we say that the change in income and the change in price will always move
  in the same direction. If the price goes up, then we have to raise income to
  keep the original bundle affordable. If the price goes down, then we have to lower income to keep the original
  bundle affordable.

  The new income to keep the same purchasing power (original bundle) is given by: \(m' = m + \Delta m\) (\(\Delta m\) maybe positive or negative).
}


\subsection{Substitution Effect}

\slide{Substitution Effect}{
  Although \((x_1,x_2)\) is still affordable, it is generally not the optimal purchase at the pivoted budget line.
  In Figure 2 below, the optimal purchase is denoted by \(Y\) on the pivoted budget line.

  \begin{center}
    \includegraphics[scale=0.3]{SE2.png} \\
    \small{Figure 2}
  \end{center}
}


\slide{Substitution Effect}{
  This bundle of goods is the optimal bundle of goods when we change the price and then adjust dollar
  income so as to keep the old bundle of goods just affordable.
  
  The movement from X to Y is known as the \textbf{substitution effect}.
  
  More precisely, the substitution effect, \(\Delta x_1^s\), is the change in the demand
  for good 1 when the price of good 1 changes to \(p_1'\) and money income changes to \(m'\):
  \[\Delta x_1^s = x_1(p_1',m') - x_1(p_1,m)\]
}


\subsection{Summary}

\slide{Substitution Effect}{
  The precise definition of the substitution effect is ``the change in the demand
  for good 1 when the price of good 1 changes to \(p_1'\) and income changes to \(m'\).''

  In essence, the substitution effect is the change in demand of good 1 when its price changes from
  \(p_1\) to \(p_1'\), keeping purchasing power (real income) constant. Then, there are two possible scenarios:

  \begin{enumerate}
    \item \textbf{Price of good 1 increases} (\(p_1' > p_1\)): If a good's price increases, then consumers
    substitute away from the good because it is relatively more expensive compared to other goods. This means
    that the demand falls \(\Rightarrow\) SE is \(-\)ve.
    \item \textbf{Price of good 1 decreases} (\(p_1' < p_1\)): If a good's price decreases, then consumers
    substitute towards the good because it is relatively cheaper compared to other goods. This means
    that the demand rises \(\Rightarrow\) SE is \(+\)ve.
  \end{enumerate}
}


\slide{Substitution Effect Simplified}{
  Substitution effect always moves opposite the price change.

  \begin{enumerate}
    \item If price \(\uparrow\), then SE is \(-\)ve.
    \item If price \(\downarrow\), then SE is \(+\)ve.
  \end{enumerate}

  This is ALWAYS true.
}


\subsection{Example}

\slide{Question}{
Suppose that John has the following demand function for milk:
\[
x_1 = 10 + \frac{M}{10 p_1}.
\]
His income is \$120 and the initial price of milk is \$3. Find the substitution effect when the price falls to \$2.
}


\slide{Solution}{
  Originally his income is \$120 and the price of milk is \$3.
  Thus his demand for milk will be: \(x_1(p_1,m) = 10 + [120/(10 \times 3)] = 14\) units of milk.

  In order to calculate the substitution effect, we must first calculate how
  much income would have to change in order to make the original bundle
  just affordable when the price of milk is \$2. We know that:
  \(\Delta m = x_1 \Delta p_1 = 14 \times (2 - 3) = -14.\)

  Therefore, the new income to keep the same purchasing power is given by:
  \(m' = m + \Delta m = 120 - 14 = 106.\)

  At this level of income and the new price of milk, John's demand for milk will be:
  \(x_1(p_1', m') = 10 + [106/(10 \times 2)] = 10 + 5.3 = 15.3\) units of milk.

  Therefore, the substitution effect is:
  \[\Delta x_1^s = x_1(p_1',m') - x_1(p_1,m) = 15.3 - 14 = 1.3\]
}