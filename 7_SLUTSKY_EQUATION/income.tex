\section{Income Effect}


\subsection{Income Effect}

\slide{Income Effect}{
  When the price of a good changes, there are two sorts of effects: the rate
  at which you can exchange one good for another changes, and the total
  purchasing power of your income is altered.

  The second part --- the change in demand due to the change in purchasing
  power --- is called the income effect.

  The second step — the shift — is a movement where the slope stays constant
  and the purchasing power changes.

  Thus the second stage of the price adjustment is called the \textbf{income effect}.
}


\slide{Income Effect}{
  We simply change the consumer's income from \(m'\) to \(m\), keeping the prices constant at \((p_1',p_2)\),
  and undoing the step in the substitution effect.

  More precisely, the income effect, \(\Delta x_1^n\), is the change in the demand for
  good 1 when we change income from \(m'\) to \(m\), holding the price of good 1
  fixed at \(p_1'\):
  \[\Delta x_1^n = x_1(p_1', m) - x_1(p_1', m')\]
}


\slide{Income Effect}{
  \begin{center}
    \includegraphics[scale=0.3]{SE2.png} \\
    \small{Figure 2}
  \end{center}

  The movement from \(Y\) to \(Z\) is the \textbf{income effect}.
}


\subsection{Summary}

\slide{Income Effect}{
  This part of the change in demand happens because a price change affects your purchasing power (real income).

  \begin{enumerate}
    \item If price of good 1 increases \(\rightarrow\) your purchasing power falls.
    \item If price of good 1 decreases \(\rightarrow\) your purchasing power rises.
  \end{enumerate}

  Real income (purchasing power) changes when price changes. How this change in real income affects demand
  depends on whether the good is normal or inferior.
}


\subsection{Normal and Inferior Goods}

\slide{Normal Good -- Price Increase}{
  \[\text{Price} \uparrow \quad \Rightarrow \quad \text{Real Income} \downarrow\]
  \begin{center}
    For a normal good, when real income falls, its demand falls.
  \end{center}
  \[\therefore \ \text{Demand} \downarrow\]
  \begin{center}
    In the case of a normal good, an increase in price causes a fall in demand. \\
    This means that IE is \(-\)ve.
  \end{center}
}


\slide{Normal Good -- Price Decrease}{
  \[\text{Price} \downarrow \quad \Rightarrow \quad \text{Real Income} \uparrow\]
  \begin{center}
    For a normal good, when real income increases, its demand increases.
  \end{center}
  \[\therefore \ \text{Demand} \uparrow\]
  \begin{center}
    In the case of a normal good, a decrease in price causes a rise in demand. \\
    This means that IE is \(+\)ve.
  \end{center}
}


\slide{Inferior Good -- Price Increase}{
  \[\text{Price} \uparrow \quad \Rightarrow \quad \text{Real Income} \downarrow\]
  \begin{center}
    For an inferior good, when real income falls, its demand increases.
  \end{center}
  \[\therefore \ \text{Demand} \uparrow\]
  \begin{center}
    In the case of an inferior good, an increase in price causes an increase in demand. \\
    This means that IE is \(+\)ve.
  \end{center}
}


\slide{Inferior Good -- Price Decrease}{
  \[\text{Price} \downarrow \quad \Rightarrow \quad \text{Real Income} \uparrow\]
  \begin{center}
    For an inferior good, when real income increases, its demand decreases.
  \end{center}
  \[\therefore \ \text{Demand} \downarrow\]
  \begin{center}
    In the case of an inferior good, a decrease in price causes a decrease in demand. \\
    This means that IE is \(-\)ve.
  \end{center}
}


\subsection{Example}

\slide{Question}{
Suppose that John has the following demand function for milk:
\[
x_1 = 10 + \frac{M}{10 p_1}.
\]
His income is \$120 and the initial price of milk is \$3. Find the income effect when the price falls to \$2.
}


\slide{Solution}{
  Firstly, we know that:
  \[\begin{aligned}
    \text{Total Change in Demand} &= \text{Substitution Effect} + \text{Income Effect} \\
    &= \Delta x_1^s + \Delta x_1^n
  \end{aligned}\]
  
  We also know that \(\Delta x_1^s = 1.3\) units of milk.

  The total change in demand when the price falls to \$2 is:
  \[x_1(2, 120) - x_1(3, 120) = \left(10 + \frac{120}{10 \cdot 2}\right) - \left(10 + \frac{120}{10 \cdot 3}\right) = 2\]
  Therefore, the income effect is:
  \[\Delta x_1^n = 2 - 1.3 = 0.7\]
}