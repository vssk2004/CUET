\section{Ordinary and Giffen Goods}


\subsection{Definitions}

\slide{Ordinary and Giffen Goods}{
    \dfn{A good \(x\) is \textbf{ordinary} if the consumer's demand for that good falls when the price of that good increases.}
    \dfn{A good \(x\) is \textbf{Giffen} if the consumer's demand for that good rises when the price of that good increases.}
}


\slide{Mathematical Definition}{
    ``A good \(x\) is \textbf{ordinary} if the consumer's demand for that good falls when the price of that good increases.''
    This means that the first derivative of the demand for good \(x\) w.r.t its price
    is negative.

    \[
    \pdv{x(p_x, p_y, m)}{p_x} < 0
    \]

    ``A good \(x\) is \textbf{Giffen} if the consumer's demand for that good rises when the price of that good increases.''
    This means that the first derivative of the demand for good \(x\) w.r.t its price
    is positive.

    \[
    \pdv{x(p_x, p_y, m)}{p_x} > 0
    \]
}


\subsection{Substitutes and Complements}

\slide{Price of the Other Good}{
    The price of the other good \(p_y\) can also affect the demand for good \(x\).
    Mathematically, two goods \(x\) and \(y\) are substitutes if:

    \[
    \pdv{x(p_x, p_y, m)}{p_y} > 0.
    \]

    The consumer desires more of good \(x\) when the price of good \(y\) increases.

    Two goods \(x\) and \(y\) are complements if:

    \[
    \pdv{x(p_x, p_y, m)}{p_y} < 0.
    \]

    The consumer desires less of good \(x\) when the price of good \(y\) increases.
}


\subsection{Demand with Different Utility Functions}

\slide{Cobb-Douglas and Change in Price}{
    \[x(p_x, p_y, m) = \frac{a}{a+b} \frac{m}{p_x}\]
    \[y(p_x, p_y, m) = \frac{b}{a+b} \frac{m}{p_y}\]

    Here, we have:
    \[
    \pdv{x(p_x, p_y, m)}{p_x} = -\frac{a}{a+b} \frac{m}{p_x^2} < 0
    \]
    \[
    \pdv{y(p_x, p_y, m)}{p_y} = -\frac{b}{a+b} \frac{m}{p_y^2} < 0
    \]

    So, both goods are ordinary.
}

\slide{Cobb-Douglas and Change in Price of Other Good}{
    \[x(p_x, p_y, m) = \frac{a}{a+b} \frac{m}{p_x}\]
    \[y(p_x, p_y, m) = \frac{b}{a+b} \frac{m}{p_y}\]

    Here, we have:
    \[
    \pdv{x(p_x, p_y, m)}{p_y} = 0
    \]
    \[
    \pdv{y(p_x, p_y, m)}{p_x} = 0
    \]

    So, the goods are neither substitutes nor complements.
}


\slide{Perfect Substitutes and Change in Price}{
    \[x = \begin{cases}
        \frac{m}{p_x} \quad \quad \ \text{when} \quad p_x < p_y \\
        [0,\frac{m}{p_x}] \quad \text{when} \quad p_x = p_y \\
        0 \quad \quad \quad \text{when} \quad p_x = p_y.
    \end{cases}\]

    If the price of good \(x\) increases, we have \(\frac{m}{\uparrow p_x}\). Therefore, \(x \downarrow\).

    If the price of good \(y\) increases, we have \(\frac{m}{\uparrow p_y}\). Therefore, \(y \downarrow\).

    Therefore, both goods are ordinary and they are substitutes.
}


\slide{Perfect Complements and Change in Price}{
    \[
    x(p_x, p_y, m) = y(p_x, p_y, m) = \frac{m}{p_x + p_y}
    \]

    If the price of good \(x\) increases, we have \(\frac{m}{\uparrow p_x + p_y}\). Therefore, \(x \downarrow\) and \(y \downarrow\).

    If the price of good \(y\) increases, we have \(\frac{m}{p_x + \uparrow p_y}\). Therefore, \(x \downarrow\) and \(y \downarrow\).

    Therefore, both goods are ordinary and they are complements.
}