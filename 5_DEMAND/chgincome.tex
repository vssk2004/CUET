\section{Demand}

\slide{Consumer Demand}{
    A consumer's optimal bundle depends on factors like income and prices.

    The Marshallian demand function captures the optimal choice of quantity for
    each good given the consumer's income \(m\) and the price of the two goods \(p_x\) and \(p_y\).
    It is given by \(x(p_x, p_y, m)\) and \(y(p_x, p_y, m)\).

    We know that the demand for a good is a function of the prices and income. Now,
    we want to know what happens to demand when prices change or income changes.

    Normal and inferior goods capture change in demand given a change in income.
    Ordinary and giffen goods capture change in demand given a change in prices.
}


\section{Normal and Inferior Goods}


\subsection{Definitions}

\slide{Normal and Inferior Goods}{
    \dfn{A good \(x\) is \textbf{normal} if the consumer's demand for that good rises when income increases.}
    \dfn{A good \(x\) is \textbf{inferior} if the consumer's demand for that good falls when income increases.}
}


\slide{Mathematical Definition}{
    ``A good \(x\) is \textbf{normal} if the consumer's demand for that good rises when income increases.''
    This means that the first derivative of the demand for good \(x\) with respect to (w.r.t) income
    is positive.

    \[
    \pdv{x(p_x, p_y, m)}{m} > 0
    \]

    ``A good \(x\) is \textbf{inferior} if the consumer's demand for that good falls when income increases.''
    This means that the first derivative of the demand for good \(x\) with respect to (w.r.t) income
    is negative.

    \[
    \pdv{x(p_x, p_y, m)}{m} < 0
    \]
}


\subsection{Normal and Inferior Goods}

\slide{Both Goods are Normal}{
    An increase in income when both goods are normal will cause the consumer's demand to rise for both goods.
    \begin{center}
        \includegraphics[scale=0.35]{NORMAL1}
    \end{center}
}

\slide{Good \texorpdfstring{\(x_1\)}{} is Inferior}{
    An increase in income when good \(x_1\) is inferior and good \(x_2\) is normal will cause the consumer's demand to fall for \(x_1\)
    and rise for \(x_2\).    
    \begin{center}
        \includegraphics[scale=0.4]{NORMAL2}
    \end{center}
}


\subsection{Demand with Different Utility Functions}

\slide{Cobb-Douglas and Change in Income}{
    The Marshallian demand given a Cobb-Douglas utility function is:
    \[x(p_x, p_y, m) = \frac{a}{a+b} \frac{m}{p_x}\]
    \[y(p_x, p_y, m) = \frac{b}{a+b} \frac{m}{p_y}\]

    From the equations, it should be evident that as \(m\) rises, so must \(x\) and \(y\).

    A consumer with preferences described by the Cobb-Douglas utility function regards
    both goods as normal.
}


\slide{Perfect Substitutes and Change in Income}{
    The Marshallian demand given the utility function \(u(x,y) = x + y\) is:
    \[x = \begin{cases}
        \frac{m}{p_x} \quad \quad \ \text{when} \quad p_x < p_y \\
        [0,\frac{m}{p_x}] \quad \text{when} \quad p_x = p_y \\
        0 \quad \quad \quad \text{when} \quad p_x = p_y.
    \end{cases}\]

    When income increases, we have \(\frac{\uparrow m}{p_x}\). Therefore, \(x \uparrow\).

    A consumer with preferences described by the Perfect Substitutes utility function regards both goods as normal.
}


\slide{Perfect Complements and Change in Income}{
    The Marshallian demand given the utility function \(u(x,y) = \min\{x,y\}\) is:
    \[x(p_x, p_y, m) = y(p_x, p_y, m) = \frac{m}{p_x + p_y}\]

    When income increases, we have \(\frac{\uparrow m}{p_x + p_y}\). Therefore, \(x \uparrow\) and \(y \uparrow\).

    A consumer with preferences described by the Perfect Complements utility function regards both goods as normal.
}