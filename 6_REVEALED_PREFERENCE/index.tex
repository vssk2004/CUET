\section{Price and Quantity Indices}


\subsection{Quantity Indices}

\slide{Quantity Index}{
  A quantity index measures how the ``average'' quantity of goods consumed changes over time.

  There are two commonly used quantity indices:
  \begin{enumerate}
    \item Laspeyres Quantity Index
    \item Paasche Quantity Index
  \end{enumerate}
}


\slide{Laspeyres Quantity Index}{
  \dfn{A \textbf{Laspeyres Quantity Index} uses base period prices for computing cost of choices
  in both base and subsequent years.}

  Let \(b\) stand for the base period, and let \(t\) be some other time. Assume that there are only two goods.
  How does “average” consumption in year \(t\) compare to consumption in the base period?

  Suppose that at time \(t\) the prices are \((p_1^t, p_2^t)\) and the consumer chooses \((x_1^t, x_2^t)\).
  In the base period \(b\) the prices are \((p_1^b, p_2^b)\) and the consumer chooses \((x_1^b, x_2^b)\).
  Then, the Laspeyres Quantity Index is:

  \[L_q = \frac{p_1^b x_1^t + p_2^b x_2^t}{p_1^b x_1^b + p_2^b x_2^b}\]
}


\slide{Paasche Quantity Index}{
  \dfn{A \textbf{Paasche Quantity Index} uses subsequent period prices for computing cost of choices
  in both base and subsequent years.}

  Let \(b\) stand for the base period, and let \(t\) be some other time. Assume that there are only two goods.
  How does “average” consumption in year \(t\) compare to consumption in the base period?

  Suppose that at time \(t\) the prices are \((p_1^t, p_2^t)\) and the consumer chooses \((x_1^t, x_2^t)\).
  In the base period \(b\) the prices are \((p_1^b, p_2^b)\) and the consumer chooses \((x_1^b, x_2^b)\).
  Then, the Paasche Quantity Index is:

  \[P_q = \frac{p_1^t x_1^t + p_2^t x_2^t}{p_1^t x_1^b + p_2^t x_2^b}\]
}


\subsection{Price Indices}

\slide{Price Index}{
  A price index measures how the cost of buying a fixed bundle changes over time.

  There are two commonly used price indices:
  \begin{enumerate}
    \item Laspeyres Price Index
    \item Paasche Price Index
  \end{enumerate}
}


\slide{Laspeyres Price Index}{
  \dfn{A \textbf{Laspeyres Price Index} compares the cost of a base period bundle 
  in both base and subsequent years.}

  Let \(b\) stand for the base period, and let \(t\) be some other time. Assume that there are only two goods.
  How does “average” price in year \(t\) compare to price in the base period?

  Suppose that at time \(t\) the prices are \((p_1^t, p_2^t)\) and the consumer chooses \((x_1^t, x_2^t)\).
  In the base period \(b\) the prices are \((p_1^b, p_2^b)\) and the consumer chooses \((x_1^b, x_2^b)\).
  Then, the Laspeyres Price Index is:

  \[L_p = \frac{p_1^t x_1^b + p_2^t x_2^b}{p_1^b x_1^b + p_2^b x_2^b}\]
}


\slide{Paasche Price Index}{
  \dfn{A \textbf{Paasche Price Index} compares the cost of a subsequent period bundle 
  in both base and subsequent years.}

  Let \(b\) stand for the base period, and let \(t\) be some other time. Assume that there are only two goods.
  How does “average” price in year \(t\) compare to price in the base period?

  Suppose that at time \(t\) the prices are \((p_1^t, p_2^t)\) and the consumer chooses \((x_1^t, x_2^t)\).
  In the base period \(b\) the prices are \((p_1^b, p_2^b)\) and the consumer chooses \((x_1^b, x_2^b)\).
  Then, the Paasche Price Index is:

  \[P_p = \frac{p_1^t x_1^t + p_2^t x_2^t}{p_1^b x_1^t + p_2^b x_2^t}\]
}


\slide{Consumer Price Index (CPI)}{
  \dfn{A measure of the cost of living. In any period, it measures the cost in
  that period of a standard basket of goods and services relative to the
  cost of the same basket of goods and services in a fixed year, called the
  base year.}

  The CPI is a Laspeyres Price Index. It is calculated as:
  \[\text{CPI}_t = \frac{\text{Cost of base-period basket in current year}}{\text{Cost of base-period basket in base year}}\]
}