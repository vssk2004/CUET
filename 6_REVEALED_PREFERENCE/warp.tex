\section{Weak Axiom of Revealed Preference (WARP)}


\subsection{Weak Axiom of Revealed Preference (WARP)}

\slide{WARP}{
  \axm{If \((x_1,x_2)\) is directly revealed preferred to \((y_1,y_2)\), and the two bundles are not the same, then
  \((y_1,y_2)\) cannot be directly revealed preferred to \((x_1,x_2)\).}

  Since \(DRP\) is a relation, it is much more helpful to think about it like this: if \(x\) is greater than \(y\) (\(x > y\)) and
  \(x\) is not equal to \(y\) (\(x \not= y\)), then \(y\) cannot be greater than \(x\) \((x \not< y)\).
}


\slide{Choices Consistent with WARP I}{
  Suppose that the blue budget line is the initial budget line, and the consumer chooses the bundle \(X\) on it. Then, the budget line
  shifts to the green one, and the consumer chooses bundle \(Y\).
  \begin{center}
    \includegraphics[scale=0.3]{DRP6}
  \end{center}
  This choice is consistent with WARP. Why?

  Initially, \(Y\) is affordable, but the consumer chooses \(X\). So, \(X\) is directly revealed preferred to \(Y\). After the budget line shifts,
  \(X\) is no longer affordable, so \(Y\) cannot be directly revealed preferred to \(X\).
}


\slide{Choices Consistent with WARP II}{
  Suppose that the blue budget line is the initial budget line, and the consumer chooses the bundle \(X\) on it. Then, the budget line
  shifts to the green one, and the consumer chooses bundle \(Y\).
  \begin{center}
    \includegraphics[scale=0.3]{DRP7}
  \end{center}
  This choice is consistent with WARP. Why?

  Initially, \(Y\) is affordable, but the consumer chooses \(X\). So, \(X\) is directly revealed preferred to \(Y\). After the budget line shifts,
  \(X\) is no longer affordable, so \(Y\) cannot be directly revealed preferred to \(X\).
}


\slide{Choices Consistent with WARP III}{
  Suppose that the blue budget line is the initial budget line, and the consumer chooses the bundle \(X\) on it. Then, the budget line
  shifts to the green one, and the consumer chooses bundle \(Y\).
  \begin{center}
    \includegraphics[scale=0.3]{DRP8}
  \end{center}
  This choice is consistent with WARP. Why?

  Initially, \(Y\) is not affordable. So, \(X\) cannot be directly revealed preferred to \(Y\). After the budget line shifts,
  \(X\) is no longer affordable, so \(Y\) cannot be directly revealed preferred to \(X\).
}


\slide{Choices Inconsistent with WARP}{
  Suppose that the blue budget line is the initial budget line, and the consumer chooses the bundle \(X\) on it. Then, the budget line
  shifts to the green one, and the consumer chooses bundle \(Y\).
  \begin{center}
    \includegraphics[scale=0.3]{DRP9}
  \end{center}
  This choice is \textbf{inconsistent} with WARP. Why?
}


\slide{Choices Inconsistent with WARP}{
  Recall that WARP states that if \(X\) is directly revealed preferred to \(Y\), and the two bundles are not the same, then
  \(Y\) cannot be directly revealed preferred to \(X\).

  Initially, \(Y\) is affordable, but the consumer chooses \(X\). So, \(X\) is directly revealed preferred to \(Y\).
  
  After the budget line shifts,
  \(X\) is still affordable, and \(Y\) is chosen, hence \(Y\) is directly revealed preferred to \(X\).
  
  This violates WARP.
}


\section{Strong Axiom of Revealed Preference (SARP)}

\slide{Strong Axiom of Revealed Preference (SARP)}{
  \axm{If \((x_1,x_2)\) is directly or indirectly revealed preferred to \((y_1,y_2)\), and the two bundles are not the same, then
  \((y_1,y_2)\) cannot be directly or indirectly revealed preferred to \((x_1,x_2)\).}
}