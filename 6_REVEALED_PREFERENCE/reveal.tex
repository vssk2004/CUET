\section{Revealed Preference}


\subsection{Idea of Revealed Preference}

\slide{Introduction to Revealed Preference}{
    Until now, we saw how we can use information about the consumer's preferences and budget constraint to
    determine his or her demand.

    Now, we reverse this process and show how we can use information
    about the consumer's demand to discover information about his or her preferences.

    In real life, preferences are not directly observable: we have to discover people's preferences from
    observing their behavior.
}


\slide{Assumptions}{
    When we talk of determining people's preferences from observing their
    behavior, we have to make a few assumptions.

    \begin{enumerate}
        \item[(1)] Monotonicity
        \item[(2)] Strict Convexity
        \item[(3)] Unchanging Preferences over Short Run
    \end{enumerate}
}


\subsection{Direct Revealed Preference}

\slide{Direct Revealed Preference}{
    Consider the following figure, where the consumer's demanded bundle \((x_1,x_2)\), and another arbitrary bundle,
    \((y_1,y_2)\), that is beneath the consumer's budget line, are shown.

    \begin{center}
        \includegraphics[scale=0.4]{DRP1}
    \end{center}
}

\slide{Direct Revealed Preference}{
    Since \((x_1,x_2)\) is the optimal bundle, it must be better than anything else that the consumer could afford.
    Hence, in particular it must be better than \((y_1,y_2)\).

    The same argument holds for any bundle on or underneath the budget line other than the demanded bundle.
    Since it could have been bought at the given budget but wasn't, then what was bought must be better.

    Here, we say that \((x_1,x_2)\) is `Directly Revealed Preferred' (DRP) to any other bundle which is affordable.

    Thus, revealed preference is a \emph{relation} that holds between the bundle that is actually demanded
    and the bundles that could have been demanded.
}


\subsection{Principle of Revealed Preference}

\slide{Principle of Revealed Preference}{
    Let \((x_1,x_2)\) be the chosen bundle when prices are \((p_1,p_2)\), and let \((y_1,y_2)\) be some other bundle such
    that \(p_1x_1 +p_2x_2 \geq p_1y_1 +p_2y_2\). Then if the consumer is choosing the most preferred bundle she can
    afford, we must have \((x_1,x_2) \succ (y_1,y_2)\).
}


\slide{Revealed Preference vs. Preference}{
    Revealed preferred means that ``X was chosen when Y was affordable.''
    
    Preference means that ``the consumer ranks X ahead of Y.''
    
    If the consumer chooses the best bundles she can afford, then “revealed preference” implies “preference.”
    Otherwise, they are two different things.

    Think of it like this; first, we have the relation \(X \; DRP \; Y\). This tells us that \(X\) is directly revealed
    preferred to \(Y\).
    Now, if we make the assumption that the consumer is rational and maximising utility, then we
    can conclude that \(X \succ Y\).
}


\subsection{Indirect Revealed Preference}

\slide{Indirect Revealed Preference}{
    Now suppose that we happen to know that \((y_1,y_2)\) is a demanded bundle
    at prices \((q_1,q_2)\) and that \((y_1,y_2)\) is itself revealed preferred to some other
    bundle \((z_1,z_2)\). That is,
    \[q_1y_1 + q_2y_2 \geq q_1z_1 + q_2z_2.\]

    Then, we know that \((x_1, x_2) \succ (y_1, y_2)\) and \((y_1, y_2) \succ (z_1, z_2)\). By transitivity of
    preferences, we can conclude that \((x_1, x_2) \succ (z_1, z_2)\).

    It is natural to say that \((x_1, x_2)\) is `Indirectly Revealed Preferred' to \((z_1, z_2)\).
}


\subsection{Recovering Preferences}

\slide{Recovering Preferences}{
    By observing choices made by the consumer, we can learn about his or her
    preferences. As we observe more and more choices, we can get a better and
    better estimate of what the consumer's preferences are like.
}


\slide{Worse Bundles}{
    If \(X = (x_1, x_2)\) is my demanded bundle, then I can say that it is better than the bundles \(Y = (y_1, y_2)\) and \(Z = (z_1, z_2)\).
    \begin{center}
        \includegraphics[scale=0.45]{DRP2}
    \end{center}
}


\slide{Better Bundles}{
\begin{columns}
    \column{0.4\textwidth}
    \centering
    \includegraphics[width=\linewidth]{DRP3}
    \small In period 1, we observe a new budget line
    causing the preferred bundle to be Y.

    \column{0.4\textwidth}
    \centering
    \includegraphics[width=\linewidth]{DRP4}
    \small In period 2, we observe yet another new budget line
    causing the preferred bundle to be Z.
\end{columns}
}

\slide{Recovering Preferences}{
    \begin{center}
        \includegraphics[scale=0.375]{DRP5}
    \end{center}
}